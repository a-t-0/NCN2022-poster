
%\vspace{-2em}
\section{Discussion}\label{sec:discussion}
%\vspace{-1em}
The reliability of the results can be improved by running the algorithm on more and larger graphs. Running on the Loihi 2 using the Lava 0.4.0 Framework may facilitate this.
\subsection*{Population Coding}\label{subsec:population_coding}
Other encoding mechanisms than sparse coding may be considered to realise radiation robustness. For example, in population coding, a population of neurons could be used to represent integer values instead of a single neuron. %The \verb+spike_once+ neuron with a synaptic output weight of $x$ could be replaced by $x$ neurons along with $m$ excitatory controller neurons that verify whether each of those neurons is still functional. If part of the population dies, the controller neurons can excite parts of the population to compensate this loss. The $m$ controller neurons could inhibit each other and form a redundancy in the redundancy mechanism.

\subsection*{Rate Coding}\label{subsec:rate_coding}
The first round of the algorithm by Alipour et al. has also been implemented using Lava V0.3.0 using a rate-coding approach, where the numbers are represented as a frequency. No radiation damage simulation has yet been performed on this implementation. However, it is expected that spike frequency modulation can be leveraged to mitigate radiation induced spike loss.

%\subsection*{Algorithm Selection}\label{subsec:algorithm_selection}
%This work has focussed on a particular optimisation problem, it can be noted that clever algorithm selection (and/or design) for radiation robust SNNs may be used to exchange approximation accuracy for robustness. For example, instead of selecting an algorithm that breaks if a single neuron dies, one could consider shortest-path algorithms that automatically yield a longer path that works around the neuron death. Furthermore, selecting applications that are closer to natural brain functionalities, such as event-based vision, may facilitate brain adaptation mechanisms at a lower cost. For example, in some deep neural networks, neuron death may be a feature instead of a bug, as the retraining phase can in some cases be used to increase the generalisability of the network. % TODO: doubt: is this an example analogous to brain adaptation?

%\subsection*{Physical Testing}\label{subsec:physical_testing}
%Many of the discussed brain adaptation implementations will fail if the boiler-plate architecture of the SNN suffer from SEEs. This issue can be resolved using fault-tolerance acceptance, redundancy and/or local shielding of boiler-plate architecture components, and by taking boiler-plate SEE propagation mechanisms into account in SNN design. The latter would require physical testing and/or detailed hardware analysis. Industry partners of the Intel Neuromorphic Research Community, such as ESA, NASA and Raytheon are also working on radiation robustness and may be able to share insight in the more detailed radiation effects on the Loihi 2 without incurring export license limitations.