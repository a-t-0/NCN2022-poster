\vspace{-2em}\section{Future Work}\label{sec:recommendations}
\vspace{-1em}
%\textit{Recommendations may be generated based on the listed issues in \cref{sec:discussion}. It is foreseeable that dedicated shielding and or redundancy of critical elements of traditional Von Neumann elements in neuromorphic architectures may be advisable, whilst relying on the brain-inspired implementations in SNNs. Another recommendation that may flow out of this research may concern the integration of brain adaptation principles into the design of SNN implementations that yield space application functionalities. Furthermore, a recommendation may be generated regarding a proposed shift in the moment of training, from pre-trained on the ground, to (re-)training after radiation exposure.}
The overarching research project aims to perform physical radiation tests to gain more insight in the practical usefulness of brain-inspired adaptation mechanisms to hedge against radiation damage in neuromorphic space hardware. Since these tests are costly, a more thorough analysis on a broader scope of adaptation mechanisms is proposed. In particular, the population coding and rate coding options will be explored.% \cref{subsec:algorithm_selection} to  \cref{subsec:physical_testing}.