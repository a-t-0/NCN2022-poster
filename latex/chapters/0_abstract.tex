%\begin{abstract}
  %Energy efficient applications of artificial intelligence (AI) may be able to increase space robot autonomy. 
  This research sets out to test whether principles of brain adaptation can be leveraged to increase the radiation resistance of neuromorphic space hardware. Neuromorphic architectures provide energy efficient platforms for AI applications in space. Structural similarities between neuromorphic architectures and the brain may allow them to benefit from brain-inspired design on the topic of damage recovery. Space environments can provide challenging conditions with significant radiation exposure, that may damage neuromorphic space hardware. To explore approaches to mitigate such damages, a simulated radiation test with- and without a brain inspired implementation adaptation is investigated. These adaptations are applied to a spiking neural network (SNN) implementation of a distributed minimum dominating set approximation algorithm by Alipour et al. The experiment is performed using Intel's Lava 0.3.0 Framework. With radiation induced neuron death probabilities of 25\%, adaptation increased SNN robustness, compared to no adaptation. At lower neuron death probabilities, no difference is observed.
 % Include audio file.
%\end{abstract}